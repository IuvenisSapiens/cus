\CUSProvideExplModule{algo}{\cus@d@te}{\cus@versi@n}{algorithms writen in TeX}

\cs_new_protected:Npn \__cus_algo_init:N #1 { \cs_set_eq:NN #1 \c_empty_tl }
\cs_new:Npn \__cus_algo_map:NN #1#2
  { 
    \exp_after:wN \__tl_map_function:Nnnnnnnnn \exp_after:wN #2 #1
      \s__tl_stop \s__tl_stop \s__tl_stop \s__tl_stop
      \s__tl_stop \s__tl_stop \s__tl_stop \s__tl_stop
    \prg_break_point:Nn \tl_map_break: { }
  }
\cs_new:Npn \__cus_algo_map:Nn #1#2
  { 
    \__cus_expanded:w { \exp_not:n { \__tl_map_tokens:nnnnnnnnn {#2} } \exp_after:wN } #1
      \s__tl_stop \s__tl_stop \s__tl_stop \s__tl_stop
      \s__tl_stop \s__tl_stop \s__tl_stop \s__tl_stop
    \prg_break_point:Nn \tl_map_break: { }
  }
\cs_new_protected:Npn \__cus_algo_put_right:Nn #1#2
  {
    \__kernel_tl_set:Nx #1
      { \__kernel_exp_not:w \exp_after:wN {#1} { \__kernel_exp_not:w {#2} } }
  }
\cs_new_protected:Npn \__cus_algo_push:Nn #1#2
  { 
    \__kernel_tl_set:Nx #1
      { { \__kernel_exp_not:w {#2} } \__kernel_exp_not:w \exp_after:wN {#1} }
  }
\cs_new_protected:Npn \__cus_algo_pop:NNTF #1#2
  {
    \if_meaning:w #1 \c_empty_tl
      \tl_set_eq:NN #2 \q_no_value
      \exp_after:wN \use_ii:nn
    \else:
      \__kernel_tl_set:Nx #2 
        { \exp_after:wN \__cus_algo_pop_aux:Nn \exp_after:wN #1 #1 } \if_false: { \fi: }
      \exp_after:wN \use_i:nn
    \fi:
  }
\cs_new_protected:Npn \__cus_algo_pop:NN #1#2
  {
    \if_meaning:w #1 \c_empty_tl
      \tl_set_eq:NN #2 \q_no_value 
    \else:
      \__kernel_tl_set:Nx #2 
        { \exp_after:wN \__cus_algo_pop_aux:Nn \exp_after:wN #1 #1 } \if_false: { \fi: }
    \fi:
  }
\cs_new:Npn \__cus_algo_pop_aux:Nn #1#2
  { 
    \__kernel_exp_not:w {#2} \if_false: { \fi: } 
    \exp_after:wN \__kernel_tl_set:Nx \exp_after:wN #1 \exp_after:wN { \if_false: } \fi:
    \__kernel_exp_not:w \exp_after:wN { \if_false: } \fi:
  }

%region 拓扑排序
% 拓扑排序是一种对有向无环图(DAG)的顶点进行线性排序的方法,使得对于每一条有向边uv,顶点u在排序中出现在v之前。
% 为了实现拓扑排序,我们可以使用以下算法:
% - 计算每个顶点的入度(指向该顶点的边的数量),并初始化已访问顶点的计数为0。
% - 选择所有入度为0的顶点,并将它们加入一个队列(入队操作)。
% - 当队列不为空时,重复以下步骤:
% -- 从队列中取出一个顶点(出队操作),并将其加入排序结果中。
% -- 增加已访问顶点的计数。
% -- 遍历该顶点的所有邻接顶点,如果邻接顶点的入度减1后变为0,则将其加入队列。
% - 如果已访问顶点的计数等于图中顶点的总数,则说明拓扑排序成功,否则说明图中存在环,无法进行拓扑排序。

\tl_new:N \l__cus_sort_a_tl 
\tl_new:N \l__cus_sort_b_tl 
\tl_new:N \g__cus_sort_result_tl 
\bool_new:N \g__cus_sort_bool 
\cs_new_protected:Npn \cus_topological_sort:nnN #1#2#3 % getter, setter, result tl 
  {
    \bool_gset_false:N \g__cus_sort_bool
    \tl_build_gbegin:N \g__cus_sort_result_tl
    \cs_set:Npn \__cus_sort_get:n ##1 {#1}
    \cs_set:Npn \__cus_sort_set:n ##1 {#2}
    \__cus_topological_sort:
    % \tl_build_get:NN \g__cus_sort_result_tl #3
    \tl_build_gend:N \g__cus_sort_result_tl
    \tl_set_eq:NN #3 \g__cus_sort_result_tl
  }
\cs_new_protected:Npn \cus_topological_sort:NNN #1#2#3
  {
    \bool_gset_false:N \g__cus_sort_bool
    \tl_build_gbegin:N \g__cus_sort_result_tl
    \cs_set_eq:NN \__cus_sort_get:n #1
    \cs_set_eq:NN \__cus_sort_set:n #2
    \__cus_topological_sort:
    % \tl_build_get:NN \g__cus_sort_result_tl #3
    \tl_build_gend:N \g__cus_sort_result_tl
    \tl_set_eq:NN #3 \g__cus_sort_result_tl
  }
\cs_new_protected:Npn \__cus_topological_sort: 
  {
    
  }
%endregion