
\begin{examcode}[label=eg:standard-toc]{这个例子展示了标准目录的输出结果。}
\tableofcontents
% 或 \standardplaincombinedlist{\contentsname}{toc}
\end{examcode}


\begin{examcode}[label=eg:standard-multitoc]{这个例子展示了多栏目录的输出方式。}
\multicolplaincombinedlist[ragged,outer-sep=0pt]{\contentsname}{toc}
\end{examcode}


\begin{examcode}[label=eg:standard-template-toc]{这个例子展示了标准的模板目录的输出结果。}
\settocdepth{section}
\templatetoc
\end{examcode}

\begin{examcode}{这个例子使用 2 栏目录,移除左侧间距,并在章标题没有编号的情况下,添加引导线,并设置超链接的文字为编号和标题。}
\makeatletter
\settocdepth{subsection}
\templatetoc [
  * = { space.left=0pt, space.hang=0pt, hyper.range={name,title} },
  chapter = { 
    code.leader={\ifx\tmcblthename\empty % 判断是否有编号
        \def\tmcbl@leadersep{4.5}%
      \else 
        \def\tmcbl@leadersep{5001}% 使用 \filler 作为引导线
      \fi 
      \tmcbl@leader@} % 默认的引导线代码
  }
] [ columns=2 ]
\makeatother
\end{examcode}

\begin{examcode}{这个例子把 \tn{section} 按行排列,只显示 \tn{chapter} 和 \tn{section} 的标题。}
% \usepackage{tabto}
\makeatletter
\templatetoc [
  * = { ignore=true }, show = {chapter,section},
  section = {
    code.before=\sbox{\@tempboxa}{\tmcblthetitle}\needhspace{\wd\@tempboxa},
    code.after=\quad, 
    code.leader=---, code.page=\tmcblthepage,
    space.left=0pt, space.right=0pt, space.hang=0pt 
  },
  chapter = { code.before=\par, space.before=1pt plus 1pt, 
    code.leader=\tabto{5cm}, % 使得页码移动到距页面左侧 5 厘米处
    code.after=\hfill\zkern\par, % 加上 \hfill\kern0pt 使得不会出现 underfull 警告
    code.name=\ifx\tmcblthename\empty
        \makebox[1em]{\rule{1ex}{1ex}}\quad
      \else
        \tmcbl@name@
      \fi 
  },
]
\makeatother
\end{examcode}