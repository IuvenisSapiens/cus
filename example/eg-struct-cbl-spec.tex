
\begin{examcode}[label=eg:specifiedtoc-structure]{这个例子展示了 \cs{SpecifiedCombinedList} 的调用结构}
\startmulticolumns[cols=3,ragged,outer-sep=0pt]

\newcommand\showthelevel{$l_{\tocthelevel}$ }
\newcommand\showhbarl{\Replicate{2*(\tocthelevel)}{│}} 
\newcommand\showhbarb{\Replicate{2*(\tocthelevel)+1}{│}} 
\newcounter{structure}
\newcounter{structurei}
\numberwithin{structurei}{structure}
\newcounter{structureii}
\numberwithin{structureii}{structurei}
% xunicode-addon 重新定义了 \textcircled,要把它的内容完全展开才能正确输出
\renewcommand*{\thestructure}{\smash{\expanded{\textcircled{\arabic{structure}}}}}
\renewcommand*{\thestructurei}{\smash{\expanded{\textcircled{\arabic{structurei}}}}}
\renewcommand*{\thestructureii}{\smash{\expanded{\textcircled{\arabic{structureii}}}}}
\newcommand\theitemindex{\stepcounter{structure\romannumeral\tocthelevel}%
  \makebox[1em]{\UseName{thestructure\romannumeral\tocthelevel}} }

\tocsetstyle{chapter,section,subsection,0,1,2}
  {\showhbarl ┌ \showthelevel list start\par}
  {\showhbarb ┌ \showthelevel block start\par}
  {\showhbarb \theitemindex \showthelevel block item\par}
  {\showhbarb └ \showthelevel block finish\par}
  {\showhbarl └ \showthelevel list finish\par}

\setlength{\parindent}{0pt}
\setlength{\lineskip}{0pt} \setlength{\lineskiplimit}{\maxdimen}
\IfFontExistsTF{TH-Times}{\fontspec{TH-Times}}{\ttfamily}\small

\specifiedtoc 

\stopmulticolumns 
\end{examcode}


\begin{examcode}[label=eg:minimal-enumitem-toc]{本例展示了使用 \pkg{enumitem} 宏包的 \env{description} 环境制作目录的一个例子。}
\startmulticolumns[ragged,outer-sep=0pt]

\newcommand{\ifinmiddle}[2]{\ifnum\tocthelevel=\tocthenextlevel\relax #1\else #2\fi}
\setlist[description,1]{nosep,leftmargin=2\ccwd}
\setlist[description,2]{nosep,leftmargin=.8\ccwd}

\tocsetstyle{chapter,section}
  {\begin{description}}
  {}
  {\item[\tocifnamed{\tocthename}{\rule{1ex}{1ex}}]
    \tocthetitle\quad\toclink{\tocthepage}\par}
  {}
  {\end{description}}
\tocsetstyle{subsection}
  {\par\begingroup\small\itshape\raggedright【\ }
  {}
  {\tocthename\enskip\tocthetitle(\toclink{\tocthepage})\ifinmiddle{;}{}}
  {}
  {】\par\endgroup\par}

\specifiedtoc

\stopmulticolumns
\end{examcode}


\begin{examcode}[label=eg:colorbox-title-toc]{本例展示了为目录添加彩色背景的方法,长标题可以换行。}
\startmulticolumns[ragged,outer-sep=0pt,column-sep=2em]

\colorlet{tocgreen}{green!65!black}
\hypersetup{hidelinks}
\newcommand{\tochyperpage}{\toclink{\tocthepage}}

\makeatletter
\tocsetstyle {chapter}
  {}
  {\noindent}
  {\fparbox{\linewidth}[padding={0pt,\fboxsep},
      border-color=tocgreen, background-color=tocgreen]
    {\bfseries\large \raggedright \color{white}%
      \hangindent4\ccwd \hangafter1 % 更推荐使用 list 环境或 description 环境 
      \strut \tocifnamed{\tocthename\unskip\quad}{}\tocthetitle
      \breakablefiller[space]\tochyperpage \strut\par }\par }
  {\smallskip}
  {}
\tocsetstyle {section}
  {\smallskip
    \begin{list}{}{\leftmargin3\ccwd \labelsep\z@ \rightmargin 2em 
      \itemindent-\ccwd \listparindent\itemindent
      \topsep\z@ \partopsep\z@ \itemsep\z@ \parsep\z@ \parskip\z@}}
  {\item \begingroup\color{tocgreen}\bfseries}
  {\tocifnamed{\tocthename\unskip\quad}{}\tocthetitle \breakablefiller[space]%
    \rlap{\makebox[2em][r]{\tochyperpage\;}}\par }
  {\endgroup}
  {\end{list}}
\tocsetstyle {subsection}
  {}
  {\begingroup\color{black}\bfseries}
  {\tocifnamed{\tocthename\unskip\quad}{}\tocthetitle \breakablefiller[dotted]%
    \rlap{\makebox[2em][r]{\tochyperpage\;}}\par }
  {\endgroup}
  {}
\makeatother 

\specifiedtoc

\stopmulticolumns
\end{examcode}


\begin{examcode}[label=eg:crazy-inline-toc]{本例展示了一个疯狂的例子。另见 \pkg{etoc} 宏包第 23 节。}
\begingroup
\newcommand\toccolorlist{red7,brown8,yellow6,olive6,teal7,cyan7,%
  azure7,blue7,magenta6,purple6}
\ExplSyntaxOn
\cs_set_nopar:Npn \toccolor #1
  {
    \tl_set:Nx \tocthecolor
      {
        \clist_item:Nn \toccolorlist
          { \int_mod:nn {#1} { \clist_count:N \toccolorlist } + 1 }
      }
    \color { \tocthecolor }
  }
\ExplSyntaxOff
\newcounter{toccolornum}
\newcommand{\ifinmiddle}[2]{\ifnum\tocthelevel=\tocthenextlevel\relax #1\else #2\fi}
\newcommand{\ifhaschild}[2]{\ifnum\tocthelevel<\tocthenextlevel\relax #1\else #2\fi}
\newcommand{\iftoout}[2]{\ifnum\tocthelevel>\tocthenextlevel\relax #1\else #2\fi}
\newcommand{\tocpageandnumber}{\,%
  \lohi{\tocifnamed{\tocthename\unskip}{}}{\toclink{P.~\tocthepage}}}

\tocsetstyle{chapter}
  {}
  {(\bgroup\stepcounter{toccolornum}\toccolor{\value{toccolornum}}}
  {{\bfseries\large \tocthetitle}\tocpageandnumber \ifhaschild{:}{}}
  {。\egroup)}
  {}
\tocsetstyle{section}
  {}
  {}
  {\textnormal{\tocthetitle}\tocpageandnumber \ifhaschild{:}{}}
  {\iftoout{}{;}}
  {}
\tocsetstyle{subsection}
  {}{}
  {\textit{\tocthetitle}\tocpageandnumber \ifinmiddle{,}{}}
  {}{}

\specifiedtoc
\endgroup
\end{examcode}


\begin{examcode}[label=eg:longtable-toc]{本例把目录放在 \env{longtable} 中。另见 \pkg{etoc} 宏包第 29 节。}
\begingroup 
\makeatletter
\newcommand{\tochangfrom}{\@hangfrom}
\makeatother
\tocsetstyle{chapter}
  {\begin{longtable}{|>{\bfseries}c|m{7cm}|r|}\hline
    \multicolumn{3}{|c|}{\Large\bfseries\strut\strut TABLE OF CONTENTS}}
  {}
  {\\\hline% 注意 \\ 必须放在 \multicolumn 紧前面,如果没有前面的 \multicolumn,则
           % 此行需改为 \tociffirst{\kill}{\\\hline}%
    \multicolumn{3}{||c||}
      {\bfseries \rule[-4ex]{0pt}{9ex}%
        \fvarbox[c]{7cm}[border-style=none]{%
          \leftskip4\ccwd \hspace*{-4\ccwd}% 在 varwidth 环境中,悬挂缩进需要一点技巧
          \tocifnamed{\tocthename\unskip\quad}{}\tocthetitle \par}}}
  {}
  {\\\hline\end{longtable}}
\tocsetstyle{section}
  {}{\\\hline} % 注意 \\ 的位置
  {\tocthename & \tocthetitle & \toclink{\tocthepage}}
  {}{}
\tocsetstyle{subsection}
  {}{\\} % 注意 \\ 的位置
  { & \tochangfrom{\tocthename\unskip\enskip}\tocthetitle & \toclink{\tocthepage}}
  {}{}
\specifiedtoc 
\endgroup
\end{examcode}


\begin{examcode}[label=eg:paracol-deco-toc]{本例展示了一个多栏目录,左侧输出垂直居中的装饰,右侧输出文字。}
\begingroup 
\hypersetup{hidelinks}
\newcommand\toccolorlist{red7,brown8,yellow6,olive6,teal7,cyan7,%
  azure7,blue7,magenta6,purple6}
\newcommand\tocornamentlist{%
  {\pgfornament[width=\columnwidth,color=\tocthecolor]{8}},%
  {\pgfornamenthan[width=\columnwidth,color=\tocthecolor]{55}},%
  {\pgfornament[width=\columnwidth,color=\tocthecolor]{67}},%
  {\pgfornament[width=\columnwidth,color=\tocthecolor]{9}},%
  {\pgfornament[width=\columnwidth,color=\tocthecolor]{4}},%
  {\pgfornament[width=\columnwidth,color=\tocthecolor]{10}},%
  {\pgfornamenthan[width=\columnwidth,color=\tocthecolor]{56}},%
  {\pgfornamenthan[width=\columnwidth,color=\tocthecolor]{57}},%
  {\pgfornament[width=\columnwidth,color=\tocthecolor]{24}},%
  {\pgfornament[width=\columnwidth,color=\tocthecolor]{1}}%
}
\ExplSyntaxOn 
\cs_set_nopar:Npn \toccolor #1
  {
    \tl_set:Nx \tocthecolor
      {
        \clist_item:Nn \toccolorlist
          { \int_mod:nn {#1} { \clist_count:N \toccolorlist } + 1 }
      }
    \color { \tocthecolor }
  }
\cs_set_nopar:Npn \tocornament #1
  {
    \clist_item:Nn \tocornamentlist 
      { \int_mod:nn {#1} { \clist_count:N \tocornamentlist } + 1 }
  }
\ExplSyntaxOff
\makeatletter
\newcounter{tochicount}
\newcommand{\tochyperpage}{\toclink{\tocthepage}}
\tocsetstyle{chapter}
  {\begingroup}
  {\stepcounter{tochicount}%
    \toccolor{\value{tochicount}}% 设置文字的颜色
    \colorlet{cusfiller}{\tocthecolor}% 设置填充的颜色
    \colseprulecolor{\tocthecolor}% 设置竖线的颜色
    \startparacol[cols=2,column-width-left={2cm/20pt},column-sep-rule=1pt]%
      \setlength{\parindent}{0pt}\nointerlineskip}
  {\vfill\makebox[\columnwidth]{\tocornament{\value{tochicount}}}% 输出左侧装饰
    \vfill\switchcolumn[1]%
% 往下的内容基本和$\text{\cref{eg:colorbox-title-toc} }$类似
    \fparbox{\columnwidth}[padding={0pt,\fboxsep},
        border-color=\tocthecolor, background-color=\tocthecolor]
      {\bfseries\Large \raggedright \color{white}%
        \hangindent4\ccwd \hangafter1 % 更推荐使用 list 环境或 description 环境 
        \strut \tocifnamed{\tocthename\unskip\quad}{}\tocthetitle
        \breakablefiller[space]\tochyperpage \strut\par }\par}
  {\stopparacol \normalcolseprulecolor \bigskip}
  {\endgroup}
\tocsetstyle {section}
  {\medskip % 这里可以再次开启一个多栏环境,注意它不能分页
    % \startmulticolumns[ragged,outer-sep=0pt,
    %   rule-width=.6pt,column-sep=1.5em,rule-color=\tocthecolor]
    \begin{list}{}{\leftmargin3\ccwd \labelsep\z@ \rightmargin 2em 
      \itemindent-3\ccwd \listparindent-\ccwd
      \topsep\z@ \partopsep\z@ \itemsep\z@ \parsep\z@ \parskip\z@}}
  {\item \begingroup\bfseries}
  {\tocifnamed{\tocthename\unskip\quad}{}\tocthetitle \breakablefiller[space]%
    \rlap{\makebox[2em][r]{\tochyperpage\;}}\par }
  {\endgroup\par 
% 这里增加了虚线的内容
    \ifnum\tocthelevel=\tocthenextlevel 
      \nobreak
      \noindent\kern-\leftmargin
      \dashfiller[.5ex]{\columnwidth}[2pt][2pt]\kern-\rightmargin \par 
    \fi}
  {\end{list}\par % \stopmulticolumns
  }
\tocsetstyle {subsection}
  {}
  {\begingroup}
  {\tocifnamed{\tocthename\unskip\quad}{}\tocthetitle \breakablefiller[dotted]%
    \rlap{\makebox[2em][r]{\tochyperpage\;}}\par }
  {\endgroup}
  {}
\makeatother
\specifiedtoc
\endgroup
\end{examcode}


\begin{examcode}[label=eg:minipage-deco-toc]{本例展示了把每个章节都放在一个盒子中,这个盒子不能分页。}
\begingroup 
\makeatletter
\hypersetup{hidelinks}

% 颜色
\newcommand\toccolorlist{red7,brown8,yellow6,olive6,teal7,cyan7,%
  azure7,blue7,magenta6,purple6}
% 左侧的装饰 
\newcommand\tocornamentlist{%
  {\pgfornament[width=\columnwidth,color=\tocthecolor]{8}},%
  {\pgfornamenthan[width=\columnwidth,color=\tocthecolor]{55}},%
  {\pgfornament[width=\columnwidth,color=\tocthecolor]{67}},%
  {\pgfornament[width=\columnwidth,color=\tocthecolor]{9}},%
  {\pgfornament[width=\columnwidth,color=\tocthecolor]{4}},%
  {\pgfornament[width=\columnwidth,color=\tocthecolor]{10}},%
  {\pgfornamenthan[width=\columnwidth,color=\tocthecolor]{56}},%
  {\pgfornamenthan[width=\columnwidth,color=\tocthecolor]{57}},%
  {\pgfornament[width=\columnwidth,color=\tocthecolor]{24}},%
  {\pgfornament[width=\columnwidth,color=\tocthecolor]{1}}%
}

\newsavebox{\tocminibox}
% 绘制外框和角落
\newcommand{\tocornament@corner}[2][han]{%
  \begin{tikzpicture}
    \node[draw,line width=.5bp,inner sep=24bp,outer sep=0pt](bx){\usebox{\tocminibox}};
    \tikzset{every node/.style={inner sep=0pt,outer sep=0pt}}
    \foreach \a/\s in {north west/none,south west/h,south east/c,north east/v}
      {\node[anchor=\a,at=(bx.\a)]
        {\UseName{pgfornament#1}[width=25bp,symmetry=\s]{\number#2}};}
  \end{tikzpicture}}

% 外框列表
\newcommand{\tocornamentcornerlist}{%
  \tocornament@corner[]{33},%
  \tocornament@corner{19},%
  \tocornament@corner{9},%
  \tocornament@corner{5},%
  \tocornament@corner[]{61},%
  \tocornament@corner{1},%
  \tocornament@corner{23},%
  \tocornament@corner[]{35},%
  \tocornament@corner[]{39},%
  \tocornament@corner{13}%
}

\ExplSyntaxOn 
\cs_set_nopar:Npn \toccolor #1
  {
    \tl_set:Nx \tocthecolor
      {
        \clist_item:Nn \toccolorlist
          { \int_mod:nn {#1} { \clist_count:N \toccolorlist } + 1 }
      }
    \color { \tocthecolor }
  }
\cs_set_nopar:Npn \tocornament #1
  {
    \clist_item:Nn \tocornamentlist 
      { \int_mod:nn {#1} { \clist_count:N \tocornamentlist } + 1 }
  }
\cs_set_nopar:Npn \tocornamentcorner #1
  {
    \clist_item:Nn \tocornamentcornerlist 
      { \int_mod:nn {#1} { \clist_count:N \tocornamentcornerlist } + 1 }
  }
\ExplSyntaxOff

% 左边的装饰
\newcommand{\tocleftdeco}{\makebox[\linewidth]{\tocornament{\value{tochicount2}}}}
\newcommand{\tocrightchapter}{%
  \fparbox{\linewidth}[padding={0pt,\fboxsep},
      border-color=\tocthecolor, background-color=\tocthecolor]
    {\bfseries\Large \raggedright \rightskip2\ccwd plus 1fil \color{white}%
      \hangindent4\ccwd \hangafter1 % 更推荐使用 list 环境或 description 环境 
      \strut \tocifnamed{\tocthename\unskip\quad}{}\tocthetitle
      \breakablefiller[space]\rlap{\makebox[2\ccwd][r]{\tochyperpage}} \strut\par }\par}

\newcounter{tochicount2}
\newcommand{\tochyperpage}{\toclink{\tocthepage}}

\tocsetstyle{chapter}
  {\begingroup}
  {\noindent \stepcounter{tochicount2}%
    \toccolor{\value{tochicount2}}% 设置文字的颜色
    \colorlet{cusfiller}{\tocthecolor}% 设置填充的颜色
    \begin{lrbox}{\tocminibox}
      % 左边和右边各用一个 minipage 
      \begin{minipage}{2cm}
        \tocleftdeco 
      \end{minipage}%
      \hspace{20pt}%
      \begin{minipage}{\dimeval{\textwidth-2cm-20pt-50bp}}% 右边盒子的宽度计算
  }
  {\tocrightchapter}
  {\end{minipage}\end{lrbox}%
    \tocornamentcorner{\value{tochicount2}}\par 
    \bigskip}
  {\endgroup}
% 下面的内容和$\text{\cref{eg:paracol-deco-toc} }$的类似
\tocsetstyle{section}
  {\bigskip 
    \startmulticolumns[ragged,outer-sep=0pt,column-sep=1.5em]
      \begin{list}{}{\leftmargin2\ccwd \labelsep\z@ \rightmargin 2em 
        \itemindent-2\ccwd \listparindent-\ccwd
        \topsep\z@ \partopsep\z@ \itemsep\z@ \parsep\z@ \parskip\z@}}
  {\item \begingroup\bfseries}
  {\tocifnamed{\tocthename\unskip\quad}{}\tocthetitle \breakablefiller[space]%
    \rlap{\makebox[2em][r]{\tochyperpage\;}}\par }
  {\endgroup\par}
  {\end{list}\par \stopmulticolumns}
\tocsetstyle{subsection}
  {}
  {\begingroup}
  {\tocifnamed{\tocthename\unskip\quad}{}\tocthetitle \breakablefiller[dotted]%
    \rlap{\makebox[2em][r]{\tochyperpage\;}}\par }
  {\endgroup}
  {}
\makeatother
\specifiedtoc
\endgroup 
\end{examcode}


\begin{examcode}[label=eg:fixed-number-width]{本例的标题数字具有固定的宽度,如果太短,则增加中间的间距,否则,压缩之。}
\makeatletter
\ekeysdeclarecmd\fixedwidthtext{smm}{\leavevmode@ifvmode
  \setbox\z@\hbox{{#3}}%
  \ifdim\dimeval{#2}>\wd\z@ 
    \hbox to\dimeval{#2}{%
      \IfBooleanTF{#1}{\xeCJKsetup{CJKglue=\hskip 0pt plus 1fill\relax,
        CJKecglue=\hskip 0pt plus 1fill\relax}%
        \spaceskip 0pt plus 1fill\relax 
        \CJKglue #3\CJKglue}{#3\hfill}}%
  \else
    \resizebox{\dimeval{#2}}{\height}{#3}%
  \fi}
\definecolor{toccol1}{HTML}{006DAA}
\definecolor{toccol2}{HTML}{C4D4E3}
\newcommand*{\zhphantom}{\vphantom{好hig}}

\tocsetstyle{chapter}{}
  {}
  {\begingroup\noindent \bfseries\large \fboxrule\z@ 
    \fcolorbox{toccol1}{toccol1}{\zhphantom\color{white}%
      \tocifnamed{\fixedwidthtext*{4\ccwd}{\tocthename\unskip}}
        {\fixedwidthtext*{4\ccwd}{\tocthetitle}}}%
    \toclinkbox{\fcolorbox{toccol2}{toccol2}{\zhphantom
      \fixedwidthtext{\linewidth-4\ccwd-4\fboxsep-\@pnumwidth}
        {\tocifnamed{\tocthetitle}{}}%
      \makebox[\@pnumwidth][r]{\tocthepage}}}
    \endgroup\par \medskip}
  {\bigskip}{}
\tocsetstyle{section}{}{}
  {\@dottedtocline{\tocthelevel}{1.5em}{2.3em}{\tocthename\enskip\tocthetitle}
    {\hss\toclink{\tocthepage}\hspace{\fboxsep}}}
  {}{}
\tocsetstyle{subsection}{}{}
  {\@dottedtocline{\tocthelevel}{3.8em}{3.2em}{\tocthename\enskip\tocthetitle}
    {\hss\toclink{\tocthepage}\hspace{\fboxsep}}}
  {}{}
\renewcommand*{\@pnumwidth}{1.3em}
\makeatother
\startmulticolumns[ragged,outer-sep=0pt,column-sep=1.5em]
\specifiedtoc 
\stopmulticolumns
\end{examcode}