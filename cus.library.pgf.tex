\CUSProvideLibrary{pgf}{\cus@d@te}{\cus@versi@n}{pgf and tikz support}

% shade text, see: https://tex.stackexchange.com/questions/192496/
\newbox\cus@picturebox 
\protected\def\shadetext#1#2{%
  \setbox\cus@picturebox=\hbox{{\cus@pdfliteral{7 Tr }#2}}%
  \tikz[baseline=0,line width=0pt]\path \pgfextra{\rlap{\copy\cus@picturebox}} [#1] 
    (0,-\dp\cus@picturebox) rectangle (\wd\cus@picturebox,\ht\cus@picturebox);}
\protected\def\shadetextbox#1{%
  \cus@collectbox\cus@picturebox\hbox{\shadetext{#1}{\unhbox\cus@picturebox}}} 

\newbox\cus@phantombox 
\protected\def\cus@phantomtobox#1{%
  \setbox\cus@phantombox=\null 
  \ht\cus@phantombox=\ht#1%
  \dp\cus@phantombox=\dp#1%
  \wd\cus@phantombox=\wd#1%
  \box\cus@phantombox}

\newcommand\shadecontent[3][]{% node options, shading options, content
  \setbox\cus@picturebox=\hbox{#3}%
  \begin{tikzpicture}[baseline=(textnode.base)]
    \node[shade,#2,inner sep=0pt,outer sep=0pt,#1](textnode)
      {\cus@phantomtobox\cus@picturebox};
    \begin{scope}[transparency group=knockout]
      \fill[white](textnode.south west)rectangle(textnode.north east);
      \node[opacity=0,inner sep=0pt,outer xsep=0pt,#1]{\box\cus@picturebox};
    \end{scope}
  \end{tikzpicture}% 
}
\newcommand\shadecontentbox[2][]{%
  \cus@collectbox\cus@picturebox\hbox{\shadecontent[{#1}]{#2}{\unhbox\cus@picturebox}}}