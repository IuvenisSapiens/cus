\CUSProvideLibrary{pgf}{\cus@d@te}{\cus@versi@n}{pgf and tikz support}
\usetikzlibrary{spath3}

% shade text, see: https://tex.stackexchange.com/questions/192496/
\newbox\cus@picturebox 
\protected\def\shadetext#1#2{%
  \setbox\cus@picturebox=\hbox{{\cus@pdfliteral{7 Tr }#2}}%
  \tikz[baseline=0,line width=0pt]\path \pgfextra{\rlap{\copy\cus@picturebox}} [#1] 
    (0,-\dp\cus@picturebox) rectangle (\wd\cus@picturebox,\ht\cus@picturebox);}
\protected\def\shadetextbox#1{%
  \@collectn@box\cus@picturebox\hbox{\shadetext{#1}{\unhbox\cus@picturebox}}} 

\newbox\cus@phantombox 
\protected\def\cus@phantomtobox#1{%
  \setbox\cus@phantombox=\null 
  \ht\cus@phantombox=\ht#1%
  \dp\cus@phantombox=\dp#1%
  \wd\cus@phantombox=\wd#1%
  \box\cus@phantombox}

\newcommand\shadecontent[3][]{% node options, shading options, content
  \setbox\cus@picturebox=\hbox{#3}%
  \begin{tikzpicture}[baseline=(textnode.base)]
    \node[shade,#2,inner sep=0pt,outer sep=0pt,#1](textnode)
      {\cus@phantomtobox\cus@picturebox};
    \begin{scope}[transparency group=knockout]
      \fill[white](textnode.south west)rectangle(textnode.north east);
      \node[opacity=0,inner sep=0pt,outer xsep=0pt,#1]{\box\cus@picturebox};
    \end{scope}
  \end{tikzpicture}% 
}
\newcommand\shadecontentbox[2][]{%
  \@collectn@box\cus@picturebox\hbox{\shadecontent[{#1}]{#2}{\unhbox\cus@picturebox}}}


\def\cus@pgf@initbfnodes{%
  \cus@pgf@initpagenode
  \cus@pgf@initlayoutnode
}
\def\cus@pgf@initpagenode{%
  \@namedef{pgf@sh@ns@page}{rectangle}%
  \@namedef{pgf@sh@nt@page}{{1}{0}{0}{1}{0pt}{0pt}}%
  \@namedef{pgf@sh@pi@page}{\pgfpictureid}%
  \@namedef{pgf@sh@np@page}{%
    \def\southwest{\pgfpointorigin}%
    \def\northeast{\pgfpoint{\paperwidth}{\paperheight}}%
  }%
}
\def\cus@pgf@initlayoutnode{%
  \@namedef{pgf@sh@ns@layout}{rectangle}%
  \@namedef{pgf@sh@nt@layout}{{1}{0}{0}{1}{0pt}{0pt}}%
  \@namedef{pgf@sh@pi@layout}{\pgfpictureid}%
  \@namedef{pgf@sh@np@layout}{%
    \def\southwest{\pgfpoint{\layoutloffset}{\the\layoutboffset}}%
    \def\northeast{\pgfpoint{\layoutloffset+\layoutwidth}{\paperheight-\layouttoffset}}%
  }%
}
\long\@namedef{__cus_bfground_tikz:n}#1{%
  \put(0pt,-\paperheight){\hb@xt@\z@{\begin{tikzpicture}
    \cus@pgf@initbfnodes
    \path[use as bounding box](page.south west)rectangle(page.north east);
    #1\end{tikzpicture}\hss}}}


\def\cus@tikz@pathend{}
\def\cus@tmp#1\tikz@path@do@at@end#2\@nil{%
  \def\tikz@finish{#1#2\cus@tikz@pathend\tikz@path@do@at@end}}
\expandafter\cus@tmp\tikz@finish\@nil
\pgfkeys{
  /tikz/tangent/.code=\@tangent@process#1\@nil,
  /tikz/at tangent/.style={shift=(tangent point #1),
    x=(tangent unit vector #1),y=(tangent orthogonal unit vector #1)}
}
\def\@tangent@process #1 at#2\@nil{\pgfkeysalso{spath/save=#1}%
  \g@addto@macro\cus@tikz@pathend{\global\let\cus@tikz@pathend\@empty
    \@tangent@getcoordinate{#1}{#1}{\@firstofone#2}}}
\def\@tangent@getcoordinate#1#2#3{\path
  [spath/use=#2,postaction=decorate,decoration={
    markings,
    mark=at position {#3}
      with
        {%
          \coordinate (tangent point #1) at (0pt,0pt);
          \coordinate (tangent unit vector #1) at (1,0pt);
          \coordinate (tangent orthogonal unit vector #1) at (0pt,1);
        }
  }];}
